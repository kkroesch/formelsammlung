% !TEX encoding = UTF-8 Unicode
\documentclass[10pt,landscape]{scrartcl}
\usepackage[left=1cm,right=1cm,top=1cm,bottom=1cm,landscape]{geometry}
\usepackage[utf8]{inputenc}
\usepackage[ngerman]{babel}
\usepackage{multicol}
\usepackage{calc}
\usepackage{amsmath,amstext,amssymb}
\usepackage{tikz}
\usepackage[europeanresistors]{circuitikz}

\usepackage[font=small,format=plain,labelfont=bf,up,labelsep=endash,textfont=it,up]{caption}

\newenvironment{Figure}
  {\par\medskip\noindent\minipage{\linewidth}}
  {\endminipage\par\medskip}

\author{kkroesch}
\title{Formelsammlung Digitaltechnik}
\begin{document}
\setlength{\columnsep}{1cm}
\begin{multicols}{3}

\noindent
Diese Formelsammlung orientiert sich an dem Prüfungsfragenkatalog des BAKOM;
die Nummerierung entspricht den Kapiteln im Katalog.
Es werden die für die Prüfung erforderlichen Formeln aufgeführt.

\section{Elektrizität}

\noindent
\fbox{\parbox[b][][t]{\columnwidth}{
	$\rho$: Spezifischer Widerstand $[\Omega~mm~m^{-1}]$\\
	$U, I, R, P$: Spannung, Strom, Widerstand, Leistung	 }}

\paragraph{Leitungswiderstand}

$$ R = \frac{\rho \cdot l}{A} $$

\paragraph{Ohmsches Gesetz}

$$ I = \frac{U}{R} = \frac{P}{U} = \sqrt{P \over R} $$
$$ R = \frac{U}{I} = \frac{U^2}{P} = \frac{P}{I^2} $$
$$ U= R \cdot I = \sqrt{P\cdot R} = \frac{P}{I} $$

\paragraph{Wechselstrom} Wechselstromgrößen werden mit Kleinbuchstaben bezeichnet.
$$ u_{SS} = \sqrt{2}\cdot u_{eff.} $$

\paragraph{Kapazitiver Blindwiderstand $X_C$}
$$ X_C =  \frac{1}{2 \pi f C} \Longleftrightarrow C =  \frac{1}{2 \pi f X_C} $$

\paragraph{Induktiver Blindwiderstand $X_L$}
$$ X_L =  2 \pi f L \Longleftrightarrow L =  \frac{X_L}{2 \pi f} $$

\section{Bauteile}

\paragraph{Spule}

\fbox{\parbox[b][][t]{\columnwidth}{
	$\tau$: Zeitkonstante \\
	$L$: Induktivität}}

$$ \tau = \frac{L}{R} $$
$$ L = \frac{U_L \cdot \Delta t}{\Delta I} $$

\noindent
\fbox{\parbox[b][][t]{\columnwidth}{
	$L$: Induktivität $[Vs A^{-1}]$			\\
	$N$: Windungszahl 					\\
	$l$: mittlere Feldlinienlänge $[m]$	\\
	$A_L$: Spulenkonstante $[Vs A^{-1}]$		\\
	$\mu_0$: magnetische Feldkonstante	\\
	$\mu_r$: relative Permeabilität}}

\paragraph{Kondensator}

\paragraph{Transformator} zur Spannungstransformation, Impedanzanpassung

\noindent
\fbox{\parbox[b][][t]{\columnwidth}{
	$N$: Windungszahl \\
	$\widehat u$: Übersetzungsverhältnis}}

\begin{Figure}
  \begin{circuitikz}
	\draw
		(0,0) node[transformer core] (T) {}
		(T.A1) node[anchor=east] {A1}
		(T.A2) node[anchor=east] {A2}
		(T.B1) node[anchor=west] {B1}
		(T.B2) node[anchor=west] {B2}
		(T.base) node{K};
  \end{circuitikz}
\end{Figure}
	
$$ \widehat{u} = \frac{N_1}{N_2} = \frac{U_1}{U_2} = \frac{I_2}{I_1} = \sqrt{\frac{R_1}{R_2}} = \sqrt{\frac{Z_1}{Z_2}}  \Longleftrightarrow Z_1 = Z_2 \cdot \widehat{u}^2 $$

Wirkungsgrad $ \eta = \frac{P _2}{P_1} 100\% $

\section{Schaltungen}

\paragraph{Zusammenschaltung} von R, L, C 

\begin{Figure}
 \centering
    \begin{circuitikz}
      \draw (0,0)
      to[R=$R$] (0,2)
      to[short] (2,2)
      to[C=$C$] (2,0)
      to[short] (0,0);
      \draw (2,2)
      to[short] (4,2)
      to[L=$L$] (4,0)
      to[short] (2,0);
   \end{circuitikz}  
   \captionof{figure}{R-L-C Parallel}
\end{Figure}

$$ Z = \sqrt{R^2 + (X_L-X_C)^2} $$

\begin{Figure}
 \centering
    \begin{circuitikz}
      \draw (0,0)
      to[short,o-] (1,0)
      to[R=$R$] (3,0)
      to[C=$C$] (5,0)
      to[L=$L$] (7,0)
      to[short,-o] (8,0);
   \end{circuitikz}  
   \captionof{figure}{R-L-C Seriell}
\end{Figure}


\paragraph{Schwingkreis} Resonanzfrequenz

\begin{Figure}
 \centering
    \begin{circuitikz}
      \draw (0,0)
      to[vsourcesin,v=$U_q$] (0,2) % The voltage source
      to[short] (2,2)
      to[C=$C$] (2,0) % The resistor
      to[short] (0,0);
      \draw (2,2)
      to[short] (4,2)
      to[L=$L$] (4,0)
      to[short] (2,0);
   \end{circuitikz}  
   \captionof{figure}{Schwingkreis}
\end{Figure}

Resonanzfrequenz
$$ f = \frac{1}{2 \pi \sqrt{L C}} $$

Bandbreite
$$ b = \frac{f_{res}}{Q} = \frac{R_v}{2 \pi L} $$

Schwingkreisgüte

$$ Q = \frac{f_{res}}{b} = \frac{f_o + f_u}{2 (f_o-f_u} = \frac{1}{d} $$
$$ Q = \frac{Z_{res}}{X_L} = \frac{Z_{res}}{X_C} = \frac{X_L}{R_V} = R_p \cdot \sqrt{C\over L} $$

\noindent
\fbox{\parbox[b][][t]{\columnwidth}{
	$R_V$: Verlustwiderstand der Spule\\
	$R_P$: LC-Parallel-Ersatzwiderstand }}

\paragraph{Transistor} Als Wechselstromverstärker

Spannungsverstärkung
$$ \beta = \frac{I_C}{I_B} $$
$$ R_C = \frac{U_B - U_{CE} - U_{RE}}{I_C} $$
$$ R_1 = \frac{U_B - U_{BE} - U_{RE}}{I_q + I_B} $$
$$ R_2 = \frac{U_{BE} + U_{RE}}{I_q} $$

\begin{Figure}
 \centering
  \begin{circuitikz}
   \ctikzset{voltage/distance from node=.2}
   \ctikzset{voltage/distance from line=.02}
   \ctikzset{voltage/bump b/.initial=.1} % Straight arrows
   \draw (4,3) node[npn](npn1) {}
    (npn1.base) node[anchor=east,yshift=2mm] {}
    (npn1.collector) node[anchor=south,xshift=2mm] {}
    (npn1.emitter) node[anchor=north,xshift=2mm] {};
   \draw (npn1.collector) to[R=$R_C$,i<=$I_C$,red] (4,6);
   \draw (0,0)
    to[short, o-] (1,0)
    to[R=$R_2$] (1,3)
    to[short,i=$I_B$,red] (npn1.base);
   \draw (0,3)
    to[C=$C_1$,o-*] (1,3);
   \draw (1,0)
    to[short,*-*] (4,0)
    to[R=$R_E$] (npn1.emitter)
    to[short,*-] (5,2.2)
    to[C=$C_E$] (5,0)
    to[short,*-*] (4,0)
    to[short,-o] (6,0);
   \draw (1,3)
    to[R=$R_2$] (1,6)
    to[short,-*] (4,6)
    to[short,-o] (6,6);
   % Voltage Arrows
   \draw [blue] (6,6) to[open,v^>=$U_B$] (6,0);
   \draw [blue] (3,3) to[open,v_>=$U_{BE}$] (3,1.5);
   \draw [blue] (3,1.5) to[open,v_>=$U_{RE}$] (3,0);
   \draw [blue] (5,4) to[open,v^>=$U_{CE}$,xshift=-2mm] (5,2);
    % Circled Transistor:
    \draw [thick] ($(npn1)-(0.16,0)$) circle [radius=16pt];
  \end{circuitikz}
  \captionof{figure}{NPN-Emitterschaltung}
\end{Figure}


\paragraph{Operationsverstärker} Text

\begin{Figure}
 \centering
  \begin{circuitikz}
   \draw (0,0) node[op amp] (opamp) {}
   (opamp.-) to [R, l_=$R_1$, *-o] ($(opamp.-)-(2,0)$) node[left]{$U_{in}$}
   (opamp.-) |- ($(opamp.-)+(0.2,1)$) to[R=$R_2$] ($(opamp.-)+(2.2,1)$) -|
   (opamp.out) to[short,*-] ($(opamp.out)+(.5,0)$) node [right] {$U_{out}$} node [ocirc] {} 
   (opamp.+) to[short]  ($(opamp.+)-(0,.5)$) node[sground] {};
  \end{circuitikz}
 \captionof{figure}{Invertierender Verstärker}
\end{Figure}

$$ V = \frac{R_2}{R_1} = \frac{U_out}{U_in} $$

\end{multicols}
\end{document}