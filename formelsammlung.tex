% !TEX encoding = UTF-8 Unicode
\documentclass[10pt,landscape]{scrartcl}
\usepackage[left=1cm,right=1cm,top=1cm,bottom=1cm,landscape]{geometry}
\usepackage[utf8]{inputenc}
\usepackage[ngerman]{babel}
\usepackage{multicol}
\usepackage{calc}
\usepackage{amsmath,amstext,amssymb}
\usepackage{tikz}
\usepackage[europeanresistors]{circuitikz}

\usepackage[font=small,format=plain,labelfont=bf,up,labelsep=endash,textfont=it,up]{caption}

\newenvironment{Figure}
  {\par\medskip\noindent\minipage{\linewidth}}
  {\endminipage\par\medskip}

\author{HB9/CEPT}
\date{}
\title{Formelsammlung}
\subtitle{f\"ur Amateurfunker}

\begin{document}
\setlength{\columnsep}{1cm}
\begin{multicols}{3}

\maketitle

\section{Elektrizität}

\noindent
\fbox{\parbox[b][][t]{\columnwidth}{
	$\rho$: Spezifischer Widerstand $[\Omega~mm~m^{-1}]$\\
	$U, I, R, P$: Spannung, Strom, Widerstand, Leistung	\\
	$E$: Feldst\"arke $[Vm^{-1}]$ }}

\subsection*{Leitungswiderstand}

$$ R = \frac{\rho \cdot l}{A} $$

\subsection*{Ohmsches Gesetz}

\begin{align*}
I =& \frac{U}{R} = \frac{P}{U} = \sqrt{P \over R} \\
R =& \frac{U}{I} = \frac{U^2}{P} = \frac{P}{I^2} \\
U =& R \cdot I = \sqrt{P\cdot R} = \frac{P}{I} \\
P =& U \cdot I = R \cdot I^2 = \frac{U^2}{R}
\end{align*}

\subsection*{Wechselstrom}

$$ u_{SS} = \sqrt{2}\cdot u_{eff.} $$

\subsection*{Kapazitiver Blindwiderstand $X_C$}
$$ X_C = \frac{1}{2 \pi f C} $$

\subsection*{Induktiver Blindwiderstand $X_L$}
$$ X_L = 2 \pi f L $$

\subsection*{Dämpfung d / Verstärkung a}

$$ 1~\textrm{Dezibel (dB)} \equiv 10 \cdot log_{10}\left(\frac{P_{1}}{P_{2}}\right) = 20~\cdot log_{10}\left(\frac{U_{1}}{U_{2}}\right) $$

\noindent
\parbox[b][6em][t]{.48\columnwidth}{
\center Verstärkung
$$ P_2 = P_1 \cdot 10^\frac{a}{10}$$
$$ U_2 = U_1 \cdot 10^\frac{a}{20} $$
}
\noindent
\parbox[b][6em][t]{.48\columnwidth}{
\center Dämpfung
$$ P_2 = P_1 \cdot 10^{-\frac{a}{10}}$$
$$ U_2 = U_1 \cdot 10^{-\frac{a}{20}} $$
}

\subsection*{Feldstärke}

$$ E = 7 \cdot \frac{\sqrt{P}}{d} $$

\section{Bauteile}

\subsection*{Spule}

$$ \tau = \frac{L}{R},\ L = U_L \cdot \frac{\Delta I}{\Delta t} $$

\noindent
\fbox{\parbox[b][][t]{\columnwidth}{
	$\tau$: Zeitkonstante \\
	$L$: Induktivität $[Vs A^{-1}]$
}}

\subsection*{Kondensator}

$$\tau = R\cdot C $$

\begin{center}
\begin{tabular}{|ll|}
\hline
Laden                   & Entladen \\
\hline
$1 \tau \approx 63,2\%$	& $1 \tau \approx 36,8\%$ \\
$2 \tau \approx 86,5\%$	& $2 \tau \approx 13,5\%$ \\
$3 \tau \approx 95,0\%$	& $3 \tau \approx 5,0\% $ \\
$4 \tau \approx 98,2\%$	& $4 \tau \approx 1,8\% $ \\
$5 \tau \approx 99,3\%$	& $5 \tau \approx 0,7\% $ \\
\hline
\end{tabular}
\end{center}

\subsection*{Transformator}

\fbox{\parbox[b][][t]{\columnwidth}{
	$N$: Windungszahl \\
	$\widehat u$: Übersetzungsverhältnis}}

$$ \widehat{u} = \frac{N_1}{N_2} = \frac{U_1}{U_2} = \frac{I_2}{I_1} = \sqrt{\frac{R_1}{R_2}} = \sqrt{\frac{Z_1}{Z_2}}  \Longleftrightarrow Z_1 = Z_2 \cdot \widehat{u}^2 $$

Wirkungsgrad $ \eta = \frac{P _2}{P_1} 100\% $

\section{Schaltungen}

\subsection*{Spannungsteiler}

\begin{Figure}
 \centering
    \begin{circuitikz}
      \ctikzset{voltage/distance from node=.2}
	  \ctikzset{voltage/distance from line=.02}
      \ctikzset{voltage/bump b/.initial=.1} % Straight arrows
      \draw (0,0)
      to[short,o-] (1,0)
      to[R=$R_2$,-*] (1,3)
      to[R=$R_1$] (1,6)
      to[short,-o,i<=$I$] (0,6);
      \draw (1,3)
      to[short,-o] (2,3);
      
      \draw [blue] (2,6)
      to[open,v^=$U_1\rightarrow R_1\cdot I$] (2,3)
      to[open,v^=$U_2\rightarrow R_2\cdot I$] (2,0);
      \draw (0,6) [blue] to[open,v=$U_{ges.}$] (0,0);
   \end{circuitikz}  
   \captionof{figure}{Spannungsteiler}
\end{Figure}

\subsection*{Reihen- und Parallelschaltung}

\noindent
\parbox[b][8em][t]{.48\columnwidth}{Reihenschaltung\\
$$R_{ges} = R_{1} + \cdots + R_{n}$$
$$\frac{1}{C_{ges}} = \frac{1}{C_{1}} + \cdots + \frac{1}{C_{n}}$$
$$L_{ges} = L_{1} + \cdots + L_{n}$$
}
\parbox[b][8em][t]{.48\columnwidth}{Parallelschaltung\\
$$\frac{1}{R_{ges}} = \frac{1}{R_{1}} + \cdots + \frac{1}{R_{n}}$$
$$C_{ges} = C_{1} + \cdots + C_{n}$$
$${L_{ges}} = \frac{1}{L_{1}} + \cdots + \frac{1}{L_{n}}$$
}

\subsection*{Schwingkreis}

\begin{Figure}
 \centering
    \begin{circuitikz}
      \draw (0,0)
      to[vsourcesin,v=$U_q$] (0,2) % The voltage source
      to[short] (2,2)
      to[C=$C$] (2,0)
      to[short] (0,0);
      \draw (2,2)
      to[short] (4,2)
      to[L=$L$] (4,0)
      to[R=$R_{L}$] (2,0);
   \end{circuitikz}  
   \captionof{figure}{Schwingkreis}
\end{Figure}

Resonanzfrequenz
$$ f = \frac{1}{2 \pi \sqrt{L C}} \Longleftrightarrow C = \frac{1}{(2\pi f_{res})^2 L} $$

Bandbreite
$$ b = \frac{f_{res}}{Q} = \frac{R_v}{2 \pi L} $$

Schwingkreisgüte
$$ Q = \frac{f_{res}}{b} = \frac{f_o + f_u}{2 (f_o-f_u} = \frac{1}{d} $$
$$ Q = \frac{Z_{res}}{X_L} = \frac{Z_{res}}{X_C} = \frac{X_L}{R_V} = R_p \cdot \sqrt{C\over L} $$

Impedanz
$$ Z = \sqrt{R^2 + (X_L-X_C)^2} $$

\begin{Figure}
 \centering
    \begin{circuitikz}
      \draw (0,0)
      to[short,o-] (1,0)
      to[R=$R$] (3,0)
      to[C=$C$] (5,0)
      to[L=$L$] (7,0)
      to[short,-o] (8,0);
   \end{circuitikz}  
\end{Figure}

Grenzfrequenz
$$ f_{grenz} = \frac{1}{2\pi} R C $$

\noindent
\fbox{\parbox[b][][t]{\columnwidth}{
	$R_V$: Verlustwiderstand der Spule\\
	$R_P$: LC-Parallel-Ersatzwiderstand }}

\subsection*{Transistor}

Spannungsverstärkung
$$ \beta = \frac{I_C}{I_B} $$
$$ R_C = \frac{U_B - U_{CE} - U_{RE}}{I_C} $$
$$ R_1 = \frac{U_B - U_{BE} - U_{RE}}{I_q + I_B} $$
$$ R_2 = \frac{U_{BE} + U_{RE}}{I_q} $$

\begin{Figure}
 \centering
  \begin{circuitikz}
   \ctikzset{voltage/distance from node=.2}
   \ctikzset{voltage/distance from line=.02}
   \ctikzset{voltage/bump b/.initial=.1} % Straight arrows
   \draw (4,3) node[npn](npn1) {}
    (npn1.base) node[anchor=east,yshift=2mm] {}
    (npn1.collector) node[anchor=south,xshift=2mm] {}
    (npn1.emitter) node[anchor=north,xshift=2mm] {};
   \draw (npn1.collector) to[R=$R_C$,i<=$I_C$,red] (4,6);
   \draw (0,0)
    to[short, o-] (1,0)
    to[R=$R_2$] (1,3)
    to[short,i=$I_B$,red] (npn1.base);
   \draw (0,3)
    to[C=$C_1$,o-*] (1,3);
   \draw (1,0)
    to[short,*-*] (4,0)
    to[R=$R_E$] (npn1.emitter)
    to[short,*-] (5,2.2)
    to[C=$C_E$] (5,0)
    to[short,*-*] (4,0)
    to[short,-o] (6,0);
   \draw (1,3)
    to[R=$R_2$] (1,6)
    to[short,-*] (4,6)
    to[short,-o] (6,6);
   % Voltage Arrows
   \draw [blue] (6,6) to[open,v^>=$U_B$] (6,0);
   \draw [blue] (3,3) to[open,v_>=$U_{BE}$] (3,1.5);
   \draw [blue] (3,1.5) to[open,v_>=$U_{RE}$] (3,0);
   \draw [blue] (5,4) to[open,v^>=$U_{CE}$,xshift=-2mm] (5,2.2);
    % Circled Transistor:
    \draw [thick] ($(npn1)-(0.16,0)$) circle [radius=16pt];
  \end{circuitikz}
  \captionof{figure}{NPN-Emitterschaltung}
\end{Figure}


\subsection*{Operationsverstärker}

\begin{Figure}
 \centering
  \begin{circuitikz}
   \draw (0,0) node[op amp] (opamp) {}
   (opamp.-) to [R, l_=$R_1$, *-o] ($(opamp.-)-(2,0)$) node[left]{$U_{in}$}
   (opamp.-) |- ($(opamp.-)+(0.2,1)$) to[R=$R_2$] ($(opamp.-)+(2.2,1)$) -|
   (opamp.out) to[short,*-] ($(opamp.out)+(.5,0)$) node [right] {$U_{out}$} node [ocirc] {} 
   (opamp.+) to[short]  ($(opamp.+)-(0,.5)$) node[sground] {};
  \end{circuitikz}
 \captionof{figure}{Invertierender Verstärker}
\end{Figure}

$$ V = \frac{R_2}{R_1} = \frac{U_{out}}{U_{in}} $$

\begin{Figure}
 \centering
  \begin{circuitikz}
   \draw (0,0) node[op amp] (opamp) {}
   (opamp.-) to [short,*-] ($(opamp.-)-(0,1)$)
   ($(opamp.-)-(0,1)$) to[R,R=$R_1$] ($(opamp.-)-(0,3)$) node[sground] {}
   (opamp.-) |- ($(opamp.-)+(0.2,1)$) to[R=$R_2$] ($(opamp.-)+(2.2,1)$) -|
   (opamp.out) to[short,*-] ($(opamp.out)+(.5,0)$) node [right] {$U_{out}$} node [ocirc] {} 
   (opamp.+) to[short,-o]  ($(opamp.+)-(2,0)$) node[left]{$U_{in}$};
  \end{circuitikz}
 \captionof{figure}{Nicht-invertierender Verstärker}
\end{Figure}

$$ V = 1 + \frac{R_{2}}{R_{1}} $$

\subsection*{Bandbreite}

\begin{align*}
b_{CW} =& \frac{5\cdot \text{WPM}}{1.2} \\
b_{AM} =& 2 \cdot f_{NFmax} \\
\end{align*}

Modulationsindex bei FM:
$$m = \frac{\Delta f}{f_{NFmax}} $$ 


\end{multicols}
\end{document}