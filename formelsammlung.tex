% !TEX encoding = UTF-8 Unicode
\documentclass[10pt,landscape]{scrartcl}
\usepackage[left=1cm,right=1cm,top=1cm,bottom=1cm,landscape]{geometry}
\usepackage[utf8]{inputenc}
\usepackage[ngerman]{babel}
\usepackage{multicol}
\usepackage{calc}
\usepackage{amsmath,amstext,amssymb}
\usepackage{tikz}
\usepackage[europeanresistors]{circuitikz}

\usepackage[font=small,format=plain,labelfont=bf,up,labelsep=endash,textfont=it,up]{caption}

\newenvironment{Figure}
  {\par\medskip\noindent\minipage{\linewidth}}
  {\endminipage\par\medskip}

\author{coltz}
\title{Formelsammlung Digitaltechnik}
\begin{document}
\setlength{\columnsep}{1cm}
\begin{multicols}{3}

\noindent
Diese Formelsammlung orientiert sich an dem Prüfungsfragenkatalog der BAKOM;
die Nummerierung entspricht den Kapiteln im Katalog.
Es werden die für die Prüfung erforderlichen Formeln aufgeführt.

\section{Elektrizität}
\paragraph{Ohmsches Gesetz}

$$ I = \frac{U}{R} $$
$$ R = \frac{U}{I} $$
$$ U= R \cdot I = \sqrt{P\cdot R} $$

\paragraph{Kapazitiver Blindwiderstand}

$$ X_C =  \frac{1}{2 \pi f C} \Longleftrightarrow C =  \frac{1}{2 \pi f X_C} $$

\paragraph{Induktiver Blindwiderstand}

$$ X_L =  2 \pi f L \Longleftrightarrow L =  \frac{X_L}{2 \pi f} $$

\section{Bauteile}

\paragraph{Spule}
$$ \tau = \frac{L}{R} $$
$$ L = \frac{U_L \cdot \Delta t}{\Delta I} $$

\noindent
\fbox{\parbox[b][][t]{\columnwidth}{
	$L$: Induktivität $[Vs A^{-1}]$			\\
	$N$: Windungszahl 					\\
	$l$: mittlere Feldlinienlänge $[m]$	\\
	$A_L$: Spulenkonstante $[Vs A^{-1}]$		\\
	$\mu_0$: magnetische Feldkonstante	\\
	$\mu_r$: relative Permeabilität}}

\paragraph{Kondensator}

\paragraph{Transformator}

$$ \widehat{u} = \frac{U_1}{U_2} = \sqrt{\frac{R_1}{R_2}} $$

\section{Schaltungen}
\paragraph{Schwingkreis}

Etwas Text

\begin{Figure}
 \centering
    \begin{circuitikz}
      \draw (0,0)
      to[vsourcesin,v=$U_q$] (0,2) % The voltage source
      to[short] (2,2)
      to[C=$C$] (2,0) % The resistor
      to[short] (0,0);
      \draw (2,2)
      to[short] (4,2)
      to[L=$L$] (4,0)
      to[short] (2,0);
%        \begin{scope}[xshift=6.5cm, yshift=.5cm]
%			\draw [->] (-2,0) -- (2.5,0) node[anchor=west] {$v_1/V$};
%			\draw [->] (0,-2) -- (0,2) node[anchor=west] {$i_1/mA$};
%			\draw (-1,0) node[anchor=north] {-2} (1,0) node[anchor=south] {2}
%				(0,1) node[anchor=west] {4} (0,-1) node[anchor=east] {-4}
%				(2,0) node[anchor=north west] {4}
%				(-1.5,0) node[anchor=south east] {-3};
%			\draw [thick] (-2,-1) -- (-1,1) -- (1,-1) -- (2,0) -- (2.5,.5);
%			\draw [dotted] (-1,1) -- (-1,0) (1,-1) -- (1,0)
%				(-1,1) -- (0,1) (1,-1) -- (0,-1);
%		\end{scope}
   \end{circuitikz}  
   \captionof{figure}{Schwingkreis}
\end{Figure}

Noch Text

\paragraph{Thomsonsche Schwingungsformel}
$$ f = \frac{1}{2 \pi \sqrt{L C}} $$

\begin{equation}
	b = \frac{f_{res}}{Q} = \frac{R_v}{2 \pi L}
\end{equation}

\paragraph{Schwingkreisgüte}

$$ Q = \frac{f_{res}}{b} = \frac{f_o + f_u}{2 (f_o-f_u} = \frac{1}{d} $$

$$ Q = \frac{Z_{res}}{X_L} = \frac{Z_{res}}{X_C} = R_p \cdot \sqrt{C\over L} $$

\paragraph{Transistor} Text

\begin{Figure}
 \centering
  \begin{circuitikz}
   \draw (0,0) node[npn](npn1) {}
    (npn1.base) node[anchor=east] {B}
    (npn1.collector) node[anchor=south] {C}
    (npn1.emitter) node[anchor=north] {E};
  \end{circuitikz}
  \captionof{figure}{NPN-Transistor}
\end{Figure}


\paragraph{Operationsverstärker} Text

\begin{Figure}
 \centering
  \begin{circuitikz}
   \draw (0,0) node[op amp] (opamp) {}
   (opamp.-) to [R, l_=$R_1$, *-o] ($(opamp.-)-(2,0)$) node[left]{$V_{in}$}
   (opamp.-) |- ($(opamp.-)+(0.2,1)$) to[R=$R_2$] ($(opamp.-)+(2.2,1)$) -|
   (opamp.out) to[short,*-] ($(opamp.out)+(.5,0)$) node [right] {$V_{out}$} node [ocirc] {} 
   (opamp.+) to[short]  ($(opamp.+)-(0,.5)$) node[sground] {}
  ;
  \end{circuitikz}
 \captionof{figure}{Invertierender Verstärker}
\end{Figure}


\end{multicols}
\end{document}